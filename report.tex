\documentclass[11pt, a4paper, titlepage, block]{article}
\usepackage{listings}

\hyphenpenalty=10000
\title{\textbf{Project for the 2013/2014 winter session\\ report}}
\author{Julian Sparber}
\date{\today}

\begin{document}

\maketitle
\section{Specifying the Problem}
	Write an ANSI C program that gets from the keyboard two propositional logic formulas and
	establishes whether they are equivalent. For the sake of simplicity, each formula can contain at
	most three occurrences of logical connectives (not, or, and, if-then, iff), which can be represented
	as single decimal digits. As a consequence, the formula can contain at most four occurrences of
	propositions, which can be represented as single letters. Each formula must not contain parentheses,
	thus it is necessary to apply precedence and associativity rules.
	\newpage
\section{Analyzing the Problem}
	\subsection{Input}
	The input are two logic formulas. The logical connectives are interpreted as numbers. \\
	\\
	\begin{tabular}{cc}
	connective & number\\
		not & 0\\
		and & 1\\
		or & 2\\
		if-then & 3\\
		iff & 4\\
	\end{tabular}\\
	\\
	Example: A1B means A $\wedge $ B\\
	\\ 
	The two formulas are inserted in the same line, with 5 as a separator.\\
	The input looks like this: A1B5B1A (A $\vee $ B $\equiv $ B $\vee $ A)\\\\
	The length is limited to 3 connectives.\\
	The output represents the equivalents of the two formulas. That means true or false.\\
	The only way to establish if two logic formulas are equivalent is to calculate the truth table of each formula and compare them.
	
	\newpage			
\section{Designing the Algorithm}
	\subsection{Input}
	The program reads two formulas character by character from the Terminal and checks for every character if it is valid.
	
	\subsection{Memory Structure}
		The formula is saved in an array with 3 elements: "name", "value" and negation ("not").\\
		There are two cases: propositions and connectives.\\
		For propositions:\\
		The "name" contains the read character. \\
		The "value" contains the state of a proposition (So 1 or 0).\\
		The "not" contains 1 if the proposition is negated and 0 when not.\\
		For connectives:\\
		The "name" contains the read character. \\
		The "value" contains the number of the connective  (So 1 or 0).\\
		The "not" is not in use
	\subsection{Elaboration}
	Every possible combination of values for the proposition is inserted into the formula, and all results are calculated.\\
	The calculation algorithm searches for the connective with the highest precedence in the formula and divides the formula into a left part and a right part. Then it carries out the same operations on each part.
	The 5 is like an "iff" if the two formulas are equal so the result is either 1 (true) or 0 (false). 
	\subsection{Output}
	The result is visualized for the user.
	\newpage
\section{Implementing the algorithm}
	See source code.

	\newpage
\section{Testing the program}
	Test formulas\\
	\\
	\begin{tabular}{|c|c|c|c|}
		\hline
			Formula & Equivalent & Program Formula & Program Equivalent \\
		\hline
			A $\vee $ B $\equiv $ B $\vee $ A & TRUE & A2B5B2A & TRUE\\
		\hline
			A $\wedge $ B $\equiv $ B $\wedge $ A & TRUE & A1B5B1A & TRUE\\
		\hline
			A $\rightarrow  $ B $\equiv $ B $\rightarrow $ A & FALSE & A3B5B3A & FALSE\\
		\hline
			A $\leftrightarrow   $ B $\equiv $ B $\leftrightarrow  $ A & TRUE & A4B5B4A & TRUE\\
		\hline
			$\neg $A $\equiv $ A & FALSE & 0A5A & FALSE\\
		\hline
			A $\wedge $ B $\wedge $  C $\wedge $ D $\wedge $  E $\equiv $ & & A1B1C1D1E5 & ERROR:\\ 

			A $\wedge $ B $\wedge $  C $\wedge $ D $\wedge $  E & TRUE & A1B1C1D1E & TO LONG FORMULA\\
		\hline
			A $\vee $ B $\wedge $ C $\equiv $ C $\wedge $ B $\vee $ A & TRUE & A2B1C5C1B2A & TRUE\\
		\hline
			$\neg $A $\vee $ B $\equiv $ A $\rightarrow  $ B& TRUE & 0A2B5A3B & TRUE\\
		\hline
			A $\vee $ B $\wedge $ C $\equiv $ C $\vee $ B $\wedge $ A & FALSE & A2B1C5C2B1A & FALSE\\
		\hline
			A $\wedge $ $\neg $A $\equiv $ A & FALSE & A10A5A & FALSE\\
		\hline
			A $\vee $ $\neg $A $\equiv $ A & FALSE & A20A5A & FALSE\\
		\hline
			$\vee $ A $\vee $ $\neg $A A $\vee $ $\equiv $ B & INVALID & 2A10AA15B & ERROR:\\
			& FORMULA &   & INVALID FORMULA\\
		\hline
	\end{tabular}

	\newpage
\section{Verifying the program}
\lstset{numbers=left, tabsize=2}
\begin{lstlisting}
{i >= 0}
int insertMem(fun mem[], char cache, int i){
     if ( i%2 == 0 ){P1}{
        mem[i].name = cache;
         mem[i].value = 0;
     }
     else{
         mem[i].name = cache;
        mem[i].value = (int)(cache - '0');
     }
    return 0;
}
\end{lstlisting}

{n ≥ 0 ∧ t = n}
r = 1; 
{P 1}
while(n ≠ 0) {P 2}{
	r = r * n;
	{P 3}
	n = n - 1;
}
{r = t!}

P 1 ≡  n ≥ 0 ∧ t = n ∧ r = 1
P 2 ≡  r = t!/n! ∧ t ≥ n ≥ 0
P 3 ≡  r = t!/(n - 1)! ∧ t ≥ n > 0

{n ≥ 0 ∧ t = n } r = 1;{ P 1 }
P 1 ⇒ P 2
{P 2 ∧ n ≠ 0 } r = r * n;{ P 3 }
{P 3} n = n - 1;{P 2}
(P 2 ∧ ¬ (n ≠ 0)) ⇒ r = t!


\end{document}
