\documentclass[11pt, a4paper, titlepage, block]{article}
\hyphenpenalty=10000
\title{\textbf{Project for the 2013/2014 winter session\\ report}}
\author{Julian Sparber}
\date{\today}

\begin{document}

\maketitle
\section{Specifying the problem}
	Write an ANSI C program that gets from the keyboard two propositional logic formulas and
	establishes whether they are equivalent. For the sake of simplicity, each formula can contain at
	most three occurrences of logical connectives (not, or, and, if-then, iff), which can be represented
	as single decimal digits. As a consequence, the formula can contain at most four occurrences of
	propositions, which can be represented as single letters. Each formula must not contain parentheses,
	thus it is necessary to apply precedence and associativity rules.
	\newpage
\section{Analyzing the problem}
	The input are two logic formulas. The logical connectives are interpreted as number. \\
	\\
	\begin{tabular}{cc}
	connective & number\\
		not & 0\\
		and & 1\\
		or & 2\\
		if-then & 3\\
		iff & 4\\
	\end{tabular}\\
	\\
	Example: A1B means A and B\\
	\\ 
	\\
	The output represent the equivalents of the two formulas. That means true or false.\\
	The only way to establish if two logic formulas are equivalent is to calculate the Truth table.
	\newpage			
\section{Designing the algorithm}
	The two formulas
	\newpage
\section{Implementing the algorithm}


	\newpage
\section{Testing the program}
	Test formulas\\
	\\
	\begin{tabular}{|c|c|c|c|}
		\hline
			A $\vee $ B $\equiv $ B $\vee $ A & TRUE & A2B5B2A & TRUE\\
		\hline
			A $\wedge $ B $\equiv $ B $\wedge $ A & TRUE & A1B5B1A & TRUE\\
		\hline
			31 & 32 & 33 & 34\\
		\hline
			41 & 42 & 43 & 44\\
		\hline
			51 & 52 & 53 & 54\\
		\hline
			61 & 62 & 63 & 64\\
		\hline
			71 & 72 & 73 & 74\\
		\hline
			81 & 82 & 83 & 84\\
		\hline
			91 & 92 & 93 & 94\\
		\hline
			101 & 102 & 103 & 104\\
		\hline
			111 & 112 & 113 & 114\\
		\hline
	\end{tabular}

	\newpage
\section{Verifying the program}

\end{document}
